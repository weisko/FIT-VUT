\documentclass[a4paper, 11pt, twocolumn] {article}

\usepackage{times}
\usepackage[czech]{babel}
\usepackage[utf8]{inputenc}
\usepackage[IL2]{fontenc}
\usepackage[left=1.5cm,text={18cm, 25cm},top=2.5cm]{geometry}

\usepackage{amsmath, amsthm, amssymb}


\begin{document}
\begin{titlepage}
\begin{center}

\theoremstyle{definition}
\newtheorem{definition} {Definice}
\newtheorem{sentence}{Věta}



{\Huge \textsc{Fakulta informačních technologií\\[0.4em]
Vysoké učení technické v~Brně}}\\[0.3em]

\vspace{\stretch{0.381966}}

{\LARGE
	Typografie a~publikování\,--\,2. projekt\\
	Sazba dokumentů a~matematických výrazů\\
}

\vspace{\stretch{0.618034}}
\end{center}

{\LARGE 2018    \hfill     Daniel Weis}
\end{titlepage}


\section*{Úvod}
V~této úloze si vyzkoušíme sazbu titulní strany, matematic\-kých
vzorců, prostředí a~dalších textových struktur obvyklých
pro technicky zaměřené texty (například rovnice (\ref{r1})
nebo Definice \ref{def1} na straně \pageref{def1}). Rovněž si vyzkoušíme používání
odkazů \verb!\ref! a~\verb!\pageref!.

Na titulní straně je využito sázení nadpisu podle optického
středu s~využitím zlatého řezu. Tento postup byl
probírán na přednášce. Dále je použito odřádkování se
zadanou relativní velikostí 0.4em a~0.3em.


\section{Matematický text}

Nejprve se podíváme na sázení matematických symbolů
a~výrazů v~plynulém textu včetně sazby definic a~vět s využitím
balíku \texttt{amsthm}. Rovněž použijeme poznámku pod
čarou s~použitím příkazu \verb!\footnote!. Někdy je vhodné
použít konstrukci \verb!${}$!, která říká, že matematický text
nemá být zalomen.


\begin{definition} \label{def1} Turingův stroj $(TS)$ \textit{je definován jako šestice
tvaru $M = (Q,\Sigma,\Gamma,\delta,q_0,q_F)$, kde}:
\end{definition}
\begin{itemize}
\item \textit {$Q$ je konečná množina} vnitřních (řídicích) stavů,
\item $\Sigma$ \textit {je konečná množina symbolů nazývaná} vstupní abeceda, $\Delta \notin \Sigma$,
\item $\Gamma$ \textit {je konečná množina symbolů}, $\Sigma \subset \Gamma$, $\Delta \in \Gamma$, \textit{nazývaná} pásková abeceda,
\item $\delta$ : $(Q\setminus\{q_F\})\times\Gamma\rightarrow Q\times(\Gamma\cup\{L,R\}),kde L,R\notin\Gamma,$ \textit{je parciální} přechodová funkce,
\item $q_0$ \textit{je počáteční stav,} $q_0 \in Q$ a
\item $q_F$ \textit{je koncový stav,} $q_F \in Q$.
\end{itemize}

Symbol $\Delta$ značí tzv. \textit{blank} (prázdný symbol), který 
se vyskytuje na místech pásky, která nebyla ještě použita
(může ale být na pásku zapsán i~později).
\par\textit{Konfigurace} \textit{pásky} se skládá z nekonečného řetězce,
který reprezentuje obsah pásky a~pozice hlavy na tomto
řetězci. Jedná se o~prvek množiny $\{\gamma\Delta^\omega\mid\gamma\in\Gamma^*\}\times\mathbb{N}$.\footnote{Pro libovolnou abecedu $\Sigma$ je $\Sigma^\omega$ množina všech\textit{ nekonečných} řetězců nad $\Sigma$,tj. nekonečných posloupností symbolů ze $\Sigma$. Pro připomenutí: $\Sigma^*$ je množina všech \textit{konečných} řetězců nad $\Sigma$.} \textit{Konfiguraci pásky} obvykle zapisujeme jako $\Delta xyz \underline{z}x\Delta \ldots$
(podtržení značí pozici hlavy). \textit{Konfigurace stroje} je pak
dána stavem řízení a~konfigurací pásky. Formálně se jedná
o~prvek množiny $Q\times\{\gamma\Delta^\omega\mid\gamma\in\Gamma^*\}\times\mathbb{N}$.

\subsection{Podsekce obsahující větu a odkaz}
\begin{definition}
\label{def:Definice2}
Řetězec $w$ nad abecedou $\Sigma$ je přijat TS \textit{$M$ jestliže $M$ při aktivaci z~počáteční konfigurace pásky $\underline{\Delta}w\Delta...$ a~počátečního stavu $q_0$ zastaví přechodem do koncového stavu $q_F$, tj. $(q_0,\Delta w\Delta^\omega,0)\underset{M}{\overset{*}{\vdash}}(q_F,\gamma,n)$ pro
nějaké $\gamma\in\Gamma^* $ a~$n \in\mathbb{N}$}.
\par\textit{Množinu $L(M) = \{w\mid w$ je přijat TS $M\} \subseteq\Sigma^*$ nazý-
váme} jazyk přijímaný TS $M$.
\end{definition}

Nyní si vyzkoušíme sazbu vět a~důkazů opět s~použitím
balíku \texttt{amsthm}.


\begin{sentence}
\textit{Třída jazyků, které jsou přijímány TS, odpovídá}
rekurzivně vyčíslitelným jazykům.
\end{sentence}
\begin{proof} 
V důkaze vyjdeme z~Definice 1 a 2.
\end{proof}

\section{Rovnice a odkazy}
Složitější matematické formulace sázíme mimo plynulý
text. Lze umístit několik výrazů na jeden řádek, ale pak je
třeba tyto vhodně oddělit, například příkazem \verb!\quad!.

$$\sqrt[i]{x_i^3}\quad\text{kde } x_i \text{ je } i\text{-té } \text{sudé číslo}\quad y^{2\cdot y_i}_i \neq y_i^{y_i^{y_i}}$$

V~rovnici (\ref{r1}) jsou využity tři typy závorek s~různou
explicitně definovanou velikostí.
\begin{eqnarray}
\label{r1}
 x & = & \bigg\{\Big(\big[a+b\big]*c\Big)^d\oplus 1\bigg\}  \\ 
 y & = & \lim\limits_{x \to \infty}\frac{\sin^2\,{x} + \cos^2\,{x}}{\frac{1}{\log_{10}x}}  \nonumber
\end{eqnarray}

\par V~této větě vidíme, jak vypadá implicitní vysázení limity $\lim_{n \to \infty}{f(n)}$ v~normálním odstavci textu. Podobně je to i s~dalšími symboly jako $\sum_{i=1}^{n} 2^i$ či $\bigcup_{A \in \mathcal{B}}$ $A$. V~případě vzorce $\lim\limits_{n \to \infty}{f(n)}$ a $\sum\limits _{i=1}^n 2^i$ jsme si vynutili méně úspornou sazbu příkazem \verb!\limits!.

\begin{eqnarray}
\int\limits_a^b f(x)\,\mathrm{d}x & = & -\int_b^a g(x)\, \mathrm{d}x\\
\overline{\overline{A \vee B}} & \Leftrightarrow & \overline{\overline{A} \vee \overline{B}}
\end{eqnarray}

\section{Matice}
Pro sázení matic se velmi často používá prostředí \texttt{array} a~závorky (\verb|\left|, \verb!\right!).

$$\left(
\begin{array}{ccc}
a + b & \widehat{\xi + \omega} & \hat{\pi}\\
\vec{a}  & \overleftrightarrow{AC} & \beta\\
\end{array} 
\right) = 1 \Longleftrightarrow \mathbb{Q} = \mathbb{R}
$$

$$
\mathbf{A} = 
\left\|
\begin{array}{cccc}
a_{11} & a_{12} & \dots & a_{1n}\\
a_{21} & a_{22} & \dots & a_{2n}\\
\vdots & \vdots & \ddots & \vdots\\
a_{m1} & a_{m2} & \dots & a_{mn}
\end{array}
\right\|= \left|
		   \begin{array}{lcr}
			t & u\\
			v~& w\\
			\end{array}
			\right|
			= tw -uv
			$$


\par Prostředí \texttt{array} lze úspěšně využít i~jinde.

$$
\binom{n}{k} =
\left\{
\begin{array}{ll}
\frac{n!}{k!(n - k)!} & \text{pro } 0 \leq k~\leq n\\
0 & \text{pro } k~< 0 \text{ nebo } k~> n
\end{array} \right.
$$

\section{Závěrem}

V~případě, že budete potřebovat vyjádřit matematickou konstrukci nebo symbol a nebude se Vám dařit jej nalézt v~samotném \LaTeX u, doporučuji prostudovat možnosti balíku maker \AmS -\LaTeX.
\end{document}
