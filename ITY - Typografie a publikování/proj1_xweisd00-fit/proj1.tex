\documentclass[a4paper, 10pt, twocolumn]{article}

\usepackage[UTF8]{inputenc}
\usepackage[czech]{babel}
\usepackage[IL2]{fontenc}
\usepackage[T1]{fontenc}

\usepackage[left=1.5cm,top=2cm,text={18cm,25cm}]{geometry}

\title{Typografie a~publikování \\1. projekt}
\author{Daniel Weis \\xweisd00@stud.fit.vutbr.cz}
\date{}

\begin{document}

\maketitle

\section{Hladká sazba}

Hladká sazba používá jeden stupeň, druh a~řez písma\linebreak
a~je sázena na stanovenou šířku odstavce. Skládá se z~odstavců
obvykle začínajících zarážkou, nejde-li o~první odstavec
za nadpisem. Mohou ale být sázeny i~bez zarážky –
rozhodující je celková grafická úprava. Odstavec končí východovou
řádkou. Věty nesmějí začínat číslicí.
\par Zvýraznění barvou, podtržením, ani změnou písma se
v~odstavcích nepoužívá. Hladká sazba je určena především
pro delší texty, jako je beletrie. Porušení konzistence sazby
působí v~textu rušivě a~unavuje čtenářův zrak.

\section{Smíšená sazba}

Smíšená sazba má o~něco volnější pravidla. Klasická hladká
sazba se doplňuje o~další řezy písma pro zvýraznění důležitých
pojmů. Existuje „pravidlo“:

\begin{quotation}
Čím více \texttt{druhů,} \textbf{\emph{řezů,}} {\small velikostí}, barev písma {\fontfamily{cmss}\selectfont a~jiných efektů} použijeme, tím \emph{profesionálněji} bude dokument vypadat. Čtenář tím {\tiny bude} vždy \textbf{\huge nadšen!}
\end{quotation}

\textsc{Tímto pravidlem se nikdy nesmíte řídit}. Příliš
časté zvýrazňování textových elementů a~změny velikosti
písma jsou známkou amatérismu autora a~působí velmi rušivě.
Dobře navržený dokument nemá obsahovat více než
4 řezy či druhy písma. Dobře navržený dokument je decentní,
ne chaotický.

\par Důležitým znakem správně vysázeného dokumentu je
konzistence-- například \textbf{tučný řez} písma bude vyhradzen
pouze pro klíčová slova, \textit{skloněný řez} pouze pro doposud
neznámé pojmy a~nebude se to míchat. Skloněný řez nepůsobí
tak rušivě a~používá se častěji. V \LaTeX{u} jej sázíme
raději příkazem \verb!\emph{text}! než \verb!\textit{text}!.

\par Smíšená sazba se nejčastěji používá pro sazbu vědeckých
článků a~technických zpráv. U~delších dokumentů
vědeckého či technického charakteru je zvykem vysvětlit
význam různých typů zvýraznění v~úvodní kapitole.

\section{Další rady:}

\begin{itemize}
\item Nadpis nesmí končit dvojtečkou a~nesmí obsahovat
odkazy na obrázky, citace, poznámky pod čarou, . . .
\item Nadpisy, číslování a~odkazy na číslované elementy
musí být sázeny příkazy k tomu určenými.
\item Výčet ani obrázek nesmí začínat hned pod nadpisem
a~nesmí tvořit celou kapitolu.
\item Poznámky pod čarou\footnote{Příliš mnoho poznámek pod čarou čtenáře zbytečně rozptyluje.} používejte opravdu střídmě.
(Šetřete i~s~textem v závorkách.)
\item Nepoužívejte velké množství malých obrázků. Zvažte,
zda je nelze seskupit.
\item Bezchybným pravopisem a~sazbou dáváme najevo
úctu ke čtenáři. Odbytý text s~chybami bude čtenář
právem považovat za nedůvěryhodný.
\end{itemize}

\section{České odlišnosti}

Česká sazba se oproti okolnímu světu v~některých aspektech
mírně liší. Jednou z odlišností je sazba uvozovek. Uvozovky
se v~češtině používají převážně pro zobrazení přímé
řeči, zvýraznění přezdívek a~ironie. V~češtině se používají
uvozovky typu \uv{9966} místo anglických “uvozovek” nebo
dvojitých "uvozovek". Lze je sázet připravenými příkazy
nebo při použití UTF-8 kódování i~přímo.

\par Ve smíšené sazbě se řez uvozovek řídí řezem prvního
uvozovaného slova. Pokud je uvozována celá věta, sází se
koncová tečka před uvozovku, pokud se uvozuje slovo nebo
část věty, sází se tečka za uvozovku.

\par Druhou odlišností je pravidlo pro sázení konců řádků.
V~české sazbě by řádek neměl končit osamocenou jednopísmennou
předložkou nebo spojkou. Spojkou \uv{a}~končit může pouze při sazbě do šířky 25 liter. Abychom \LaTeX{u}
zabránili v~sázení osamocených předložek, spojujeme je s~následujícím slovem {\fontfamily{pzc}\selectfont nezlomitelnou mezerou}. Tu sázíme pomocí znaku \verb|~| (vlnka, tilda). Pro systematické doplnění vlnek slouží volně šiřitelný program {\fontfamily{pzc}\selectfont vlna} od pana Olšáka\footnote{{\fontfamily{cmss}\selectfont Viz http://petr.olsak.net/ftp/olsak/vlna/}}.

\par Balíček {\fontfamily{cmss}\selectfont fontenc} slouží ke korektnímu kódovaní znaků s~diakritikou, aby bylo možno v~textu vyhledávat a~kopírovat z~nej.

\section{Závěr}

Tento dokument schválně obsahuje několik typografických
prohřešků. Sekce 2 a 3 obsahují typografické chyby. V kontextu
celého textu je jistě snadno najdete. Je dobré znát
možnosti \LaTeX{u}, ale je také nutné vědět, kdy je nepoužít.

\end{document}